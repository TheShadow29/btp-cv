\documentclass{article}
\usepackage[a4paper, tmargin=1in, bmargin=1in]{geometry}
\usepackage[utf8]{inputenc}
\usepackage{graphicx}
\usepackage{parskip}
\usepackage{pdflscape}
\usepackage{listings}
\usepackage{hyperref}
\usepackage{float}
\usepackage{caption}
\usepackage{subcaption}
\usepackage{amssymb}

% \usepackage{siunitx}
% \sisetup{round-mode=places,round-precision=2}

\newcommand{\ra}{$\rightarrow$}


\title{Paper Summaries}
\author{
  Arka Sadhu}
\date{\today}

\begin{document}
\maketitle

\tableofcontents
\newpage

\section{QRS Analysis Algos}
\subsection{Signal Derivatives and Digital Filters}
\begin{itemize}
\item Typical f components in the range of 10-25 Hz, so most algos use a filter bank to attenuate other signal artifacts from P,T waves, baseline drift, incoupling noise.
\item For P,T waves use high pass filtering, for incoupling noise actually need a low pass filter, giving a bandpass filter from 10-25 Hz.
\item Some do it separately , some only take the high pass part of it.
\item Most algos use some kind of decision rules to reduce the number of false positives.
\end{itemize}
\subsubsection{Derivative Based Algos}
\begin{itemize}
\item HPF realized as a differentiator. Mostly first order, some also second order. Some cases a linear combination as well.
\item Detection by comparing the feature against a threshold.
\item Also complemented by heuristically found features
\end{itemize}
\subsubsection{Digital Filters}
\begin{itemize}
\item Two different lpf with different cut-off freq, subtraction gives bpf.
\item Passed onto simple m+-time step averaging.
\item MOBD (multiplicatio of backward difference) : kind of AND all algorithm  and some consistency conditions.
\item Simple peak detection search by comparing the max and following till v/2 of the signal. Mark the highest peak.
\item Do a form of lc of peak level and simple noise level and update, eventually threshold reached.
\item Another method : max of each segment compared to an adaptive noise level and adaptive peak estimate and classified depending on distance to each other.
\item Generalized digital filters also proposed.
\end{itemize}

\subsection{Wavelet, Singularity, Filter Bank}
\begin{itemize}
\item Use wavelet transform, in some sense similar to stft. Use discrete wavelet transform.
\item 32 band filter bank used to downsample subband signals.
\end{itemize}

\subsection{Neural Network Based Methods}

\end{document}