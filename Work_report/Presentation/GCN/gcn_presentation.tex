\documentclass{beamer}
\usetheme{Boadilla}
% \usepackage[a4paper, tmargin=1in, bmargin=1in]{geometry}
\usepackage[utf8]{inputenc}
\usepackage{graphicx}
\usepackage{parskip}
\usepackage{pdflscape}
\usepackage{listings}
\usepackage{hyperref}
\usepackage{float}
\usepackage{caption}
\usepackage{subcaption}
\usepackage{amssymb}


\title{Graph Convolution Networks}
% \subtitle{Using Beamer}
\author{Arka Sadhu}
\institute{IIT Bombay}
\date{\today}

\begin{document}
% document goes here


\begin{frame}
\titlepage
\end{frame}

\begin{frame}
\frametitle{Outline}
\tableofcontents
\end{frame}

\section{Graph Convolutional Networks}
\subsection{Why GCN}
\begin{frame}
  \frametitle{Introduction to Graph Convolutional Networks}
  \begin{itemize}
  \item <1-> CNN are extremely efficient architectures for image and audio classification tasks.
  \item <1-> But CNN donot directly generalize to irregular domains such as graph.
  \item <2-> Want to generalize CNN to Graphs.
  % \item <3-> Some applications would be spatio-temporal
  \end{itemize}
\end{frame}

\subsection{How to extend convolution to graphs?}
\begin{frame}
  \frametitle{Extending Convolutional to Graphs}
  There are two main approaches
  \begin{itemize}
  \item <1->Spatial Approach :\\
    Generalization of CNN in the spatial domain itself.
    \begin{itemize}
    \item <2-> Learning Convolutional Neural Networks for Graphs [ICML 2016].\cite{DBLP:journals/corr/NiepertAK16}
    \end{itemize}
  \item <3-> Spectral Approach :\\
    Using the frequency characterization of CNN and using that to generalize to Graphical domain
    \begin{itemize}
    \item <4->Spectral Networks and Deep Locally Connected Networks on Graphs [ICLR 2014]. % \cite{DBLP:journals/corr/BrunaZSL13} %
    \item <4->Convolutional Neural Networks on Graphs with Fast Localized Spectral Filtering [NIPS 2016] % \cite{DBLP:journals/corr/DefferrardBV16}
    \item <4->Semi-Supervised Classification with Graph Convolutional Networks [ICLR 2017] % \cite{DBLP:journals/corr/KipfW16}
    \end{itemize}
  \end{itemize}
\end{frame}

\section{Spatial Approach}
\begin{frame}
  \frametitle{Direct extension of convolution}

\end{frame}

% \section{Section 1}
% \subsection{sub a}

% \begin{frame}
% \frametitle{Title}
% Lorem ipsum dolor sit amet, consectetur adipisicing elit, sed do eiusmod tempor incididunt ut labore et dolore magna aliqua.
% \end{frame}

% \section{sub b}
% \begin{frame}
% \frametitle{List}
% \begin{itemize}
% \item<1-> Point A
% \item<2-> Point B
% \begin{itemize}
% \item part 1
% \item part 2
% \end{itemize}
% \item <3->Point C
% \item <4->Point D
% \end{itemize}
% \end{frame}

% \section{section 2}
% \begin{frame}
%   \frametitle{Enumerate}
%   \begin{enumerate}[I]
%   \item Point A
%   \item Point B
%     \begin{enumerate}[i]
%     \item part 1
%     \item part 2
%     \end{enumerate}
%   \item Point C
%   \item Point D
%   \end{enumerate}

% \end{frame}

% \subsection{sub A}
% \begin{frame}
% \frametitle{Using Columns}
% \begin{columns}
% \column{0.5\textwidth}
% \begin{description}
% \item[API] Application Programming Interface
% \item[LAN] Local Area Network
% \item[ASCII] American Standard Code for Information Interchange
% \end{description}

% \column{0.5\textwidth}
% \begin{description}
% \item[API] Application Programming Interface
% \item[LAN] Local Area Network
% \item[ASCII] American Standard Code for Information Interchange
% \end{description}

% \end{columns}
% \end{frame}

\bibliography{ref.bib}
\bibliographystyle{ieeetr}


\end{document}
