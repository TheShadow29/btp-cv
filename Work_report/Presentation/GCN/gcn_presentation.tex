\documentclass{beamer}
\usetheme{Boadilla}
% \usepackage[a4paper, tmargin=1in, bmargin=1in]{geometry}
\usepackage[utf8]{inputenc}
\usepackage{graphicx}
\usepackage{parskip}
\usepackage{pdflscape}
\usepackage{listings}
\usepackage{hyperref}
\usepackage{float}
\usepackage{caption}
\usepackage{subcaption}
\usepackage{amssymb}


\AtBeginSubsection[
  {\frame<beamer>{\frametitle{Outline}
    \tableofcontents[currentsection,currentsubsection]}}%
]%
{
  \frame<beamer>{
    \frametitle{Outline}
    \tableofcontents[currentsection,currentsubsection]}
}

\title{Graph Convolution Networks}
% \subtitle{Using Beamer}
\author{Arka Sadhu}
\institute{IIT Bombay}
\date{\today}


\begin{document}
% document goes here


\begin{frame}
\titlepage
\end{frame}

\begin{frame}
\frametitle{Outline}
\tableofcontents
\end{frame}

\section{Graph Convolutional Networks}

\subsection{Why GCN}
\begin{frame}
  \frametitle{Introduction to Graph Convolutional Networks}
  \begin{itemize}
  \item <1-> CNN are extremely efficient architectures for image and audio classification tasks.
  \item <1-> But CNN donot directly generalize to irregular domains such as graph.
  \item <2-> Want to generalize CNN to Graphs.
  \item <3-> Non-trivial because the distances are non-euclidean.
  % \item <3-> Some applications would be spatio-temporal
  \end{itemize}
\end{frame}

\subsection{How to extend convolution to graphs?}
\begin{frame}
  \frametitle{Extending Convolutional to Graphs}
  There are two main approaches
  \begin{itemize}
  \item <1->Spatial Approach :\\
    Generalization of CNN in the spatial domain itself.
    \begin{itemize}
    \item <2-> Learning Convolutional Neural Networks for Graphs [ICML 2016].\cite{DBLP:journals/corr/NiepertAK16}
    \end{itemize}
  \item <3-> Spectral Approach :\\
    Using the frequency characterization of CNN and using that to generalize to Graphical domain
    \begin{itemize}
    \item <4->Spectral Networks and Deep Locally Connected Networks on Graphs [Bruna et al. ICLR 2014]. % \cite{DBLP:journals/corr/BrunaZSL13} %
    \item <4->Convolutional Neural Networks on Graphs with Fast Localized Spectral Filtering [Defferrard et al. NIPS 2016] (will be the main focus) % \cite{DBLP:journals/corr/DefferrardBV16}
    \item <4->Semi-Supervised Classification with Graph Convolutional Networks [Kipf et al. ICLR 2017] % \cite{DBLP:journals/corr/KipfW16}
    \end{itemize}
  \end{itemize}
\end{frame}

\section{Spatial Approach}
\begin{frame}
  \frametitle{Limitations of Spatial Approach}
  \begin{itemize}
  \item Can't exactly define a neighborhood because the distances are not uniform.
  \item Ordering of nodes is problem specific.
  \end{itemize}
  Hence for the remainder we discuss the Spectral Approach
\end{frame}

\section{Spectral Approach}
\subsection{Basics of Spectral Approach}
\begin{frame}
  \frametitle{A Basic Formulation}
  \begin{itemize}
  \item Convolution in spectral (Fourier) domain is point wise multiplication.
  \item Fourier Basis is defined as the eigen basis of the laplacian operator.
  \item Can use Laplacian of a graph.
  \end{itemize}
\end{frame}

\subsection{Problem Formulation}
\begin{frame}
  \frametitle{Defining the Problem on Graphs}
  \begin{itemize}
  \item <1-> A feature description $x_i$ for every node $i$; summarized in a $N x D$ feature matrix $X$ ($N$ : number of nodes, $D$ : number of input features)
  \item <2-> Adjacency Matrix $A$.
  \item <3-> Node level output $Z$ (an $N x F$ feature matrix, where $F$ = number of output features per node).
  \end{itemize}
\end{frame}

\subsection{Graph Laplacian}
\begin{frame}
  \frametitle{Brief overview of Graph Laplacian}
  Let $T$ denote the diagonal matrix with ($v$,$v$)-th entry having value $d_v$ : degree of vertex $v$.
  % $$L(u,v) = $$
  Define L-matrix as
  \[
    L(u,v) =
    \begin{cases}
      d_v &\quad\text{if u = v}\\
      -1 &\quad\text{if u and v are adjacent}\\
      0 &\quad \text{otherwise}\\
    \end{cases}
  \]
  And the Laplacian of the graph as
  \[
    \mathcal{L}(u,v) =
    \begin{cases}
      1 &\quad\text{if u = v}\\
      -\frac{1}{\sqrt{d_ud_v}} &\quad\text{if u and v are adjacent}\\
      0 &\quad \text{otherwise}\\
    \end{cases}
  \]
\end{frame}

\begin{frame}
  \frametitle{Graph Laplacian (contd.)}
  $$\mathcal{L} = T^{-1/2}LT^{1/2}$$
  With the convention $T^{-1}(v,v) = 0$ for $d_v = 0$.\\
  When G is k-regular,
  $$\mathcal{L} = I - \frac{1}{k}A$$
  For a general graph
  $$\mathcal{L} = I - T^{-1/2}AT^{1/2}$$
\end{frame}

\section{CNN on Graphs with Fast Localized Spectral Filtering}
\subsection{Learning fast localized Spectral filters}
\begin{frame}
  \frametitle{Graph Fourier Transform}
  \begin{itemize}
  \item Laplcian of the graph is real symmetric positive semidefinite, and thus can be written as
    $$L = U \Lambda U^{T}$$
  \item Here $U = [u_0 .... u_{n-1}]$ is the fourier basis and $\Lambda = diag([\lambda_0...\lambda_{n-1}])$ are ordered real non-negative eigen values.
  \item Graph Fourier Transform of a signal $x$ is $\hat{x} = U^{T}x$.
  \end{itemize}
\end{frame}

\begin{frame}
  \frametitle{Spectral filtering of graph signals}
  \begin{itemize}
  \item Defining convolution on graphs
    $$ x *_{G} y = U((U^T x) \odot (U^T y))$$
  \item Filtering by $g_{\theta}$
    $$ y = g_{\theta}(L)x = g_{\theta}(U \Lambda U^{T})x = Ug_{\theta}(\Lambda) U^T x $$
  \item A non-parametric filter (all parameters free) would be defined as
    $$g_{\theta}(\Lambda) = diag(\theta)$$
  \end{itemize}
\end{frame}

\begin{frame}
  \frametitle{Polynomial Parametrization}
  \begin{itemize}
  \item   Problem with non-parametric filters is that not localized (we want something like k-neighborhood) and therefore their learning complexity becomes $O(n)$. This can be overcomed with use of a Polynomial filter
    $$g_{\theta}(\Lambda) = \sum_{k = 0}^{K-1}\theta_k\Lambda^k$$
  \item The advantage we gain here is that nodes which are at a distance greater than $K$ away from the node $i$, at which the filter is applied, are not affected. Hence we have gained localization.
  \end{itemize}
\end{frame}

\begin{frame}
  \frametitle{Recursive formulation for fast filtering}
  \begin{itemize}
  \item Still cost to filter is high $O(n^2)$ because of multiplication with $U$ matrix.
  \item Therefore use recurrence relation of chebyshev polynomial instead.
    $$g_{\theta}(\Lambda) = \sum_{k=0}^{K-1}\theta_k T_K(\tilde{\Lambda})$$
    Here $\tilde{\Lambda}$ is scaled between $[-1,1]$.
  \item This allows us to compute $\bar{x_k} = T_K{\tilde{L}}x$. And Therefore
    $$ y = g_{\theta}(L)x = [\bar{x_0}...\bar{x_{k-1}}]\theta$$
  \item The cost is now $O(K|E|)$
  \end{itemize}
\end{frame}

\begin{frame}
  \frametitle{Learning filters}
  \begin{itemize}
  \item Trivial to show that backprop calculation can be done efficiently.
  \end{itemize}
\end{frame}

\bibliography{ref.bib}
\bibliographystyle{ieeetr}


\end{document}
